\documentclass{article}
\usepackage{array}
\usepackage{amsfonts}
\usepackage{amsmath}
\usepackage{geometry}
\usepackage{stmaryrd}
\geometry{hmargin=2.5cm,vmargin=1.5cm}

\title{Homework 1}
\author{Arnaud Minondo}

\begin{document}
    

    \maketitle


    \section*{Exercise 1}
    \subsection*{1.a}
    The probability that ``a chosen family has i children"(event A) is : 
    \\ 
    $\mathbb{P}(A_{i}) = \frac{\textrm{number of family having i children}}{\textrm{total number of family}}$ which yields the following table 
    \begin{tabular}{|c|c|c|c|c|c|}
        \hline
        i & 1 & 2 & 3 & 4 & 5 \\
        \hline
        $\mathbb{P}(A_{i})$ & $\frac{1}{5}$ & $\frac{2}{5}$ & $\frac{1}{4}$ & $\frac{1}{10}$ & $\frac{1}{20}$
        \\
        \hline
    \end{tabular}   
    \subsection*{1.b}
    The same formula applies for the children : $P(B_i) = \frac{\textrm{number of children coming from a family of i children}}{\textrm{total number of children}}$
    \\
    It yields the following table :
    \begin{tabular}{|c|c|c|c|c|c|}
        \hline
        i & 1 & 2 & 3 & 4 & 5 \\
        \hline
        $\mathbb{P}(B_i)$ & $\frac{1}{12}$ & $\frac{1}{3}$ & $\frac{5}{16}$ & $\frac{1}{6}$ & $\frac{5}{48}$ 
        \\
        \hline
    \end{tabular} 
    \section*{Exercise 2}
    The chance of ``obtaining a double 6"(event D) when throwing two dice is : $\mathbb{P}(D) = \frac{1}{36}$ 
    \\
    The probability of throwing the two dice n times and never observing T is : $C_n = \bar{D}\bar{D}... \bar{D}$ n times. 
    \\
    Thus : $\mathbb{P}(C_n) = P(\bar{T})^n = (1-\frac{1}{36})^n$ as each throw is independent from the other. 
    \\
    Hence the probability of observing at least a double 6 is : $\boxed{\mathbb{P}(\bar{C_n}) = 1 - (\frac{35}{36})^n }$
    \\
    For this probability to be over $\frac{1}{2}$ it requires $ n \geq 25 $.

    \section*{Exercise 3}
    I will denote the event : ``at least one ball of each color is chosen" as A.
    \\
    So $\bar{A} = $ ``two colors or less are chosen".
    \\
    Chosing at most two colors means chosing two colors and taking balls from
     those two
    ; which means either choosing white and red : $\binom{11}{5}$ ; red and blue 
    : $\binom{12}{5}$ ; blue and white : $\binom{13}{5}$
    \\
    In these binomial coefficient we count the possibility of choosing only one color
     for the two colors selected which means in the sum we need to remove once the possibilities
     of choosing only one color : for red $\binom{5}{5}$ ; for white $\binom{6}{5}$ ; for blue $\binom{7}{5}$
    \\
    The total number of possibilities of finding only two colors or less is : $\binom{13}{5} + \binom{12}{5} + 
    \binom{11}{5} - \binom{7}{5} - \binom{6}{5} - \binom{5}{5} $
    \\
    There is in total $\binom{18}{5}$ possibilities of chosing 5 balls among the 18 available in the urn.
    \\
    Hence : $\boxed{\mathbb{P}(A) = \frac{\binom{18}{5}-(\binom{13}{5} + \binom{12}{5} + 
    \binom{11}{5} - \binom{7}{5} - \binom{6}{5} - \binom{5}{5}) }
    {\binom{18}{5}} \approx 0,706699}$\\
    PS : this is also equal to $\frac{\binom{5}{2}\binom{6}{2}\binom{7}{1}+\binom{6}{2}\binom{5}{1}\binom{7}{2}+\binom{5}{2}\binom{6}{1}\binom{7}{2}+\binom{7}{3}\binom{6}{1}\binom{5}{1}+\binom{7}{1}\binom{6}{3}\binom{5}{1}+\binom{7}{1}\binom{6}{1}\binom{5}{3}}{\binom{18}{5}}$
    \section*{Exercise 4}
    \subsection*{4.1}
    I will denote : $B_i$ = ``the i-th child is a boy''
    \\
    So that $A = B_1B_2B_3\cup\bar{B_1}\bar{B_2}\bar{B_3}$
    \\
    And $B =\cup_{i\in\llbracket1;3\rrbracket}B_i$ 
    \\
    And $C = (\cup_{i\in\llbracket1;3\rrbracket}B_i)\cap(\cup_{i\in\llbracket1;3\rrbracket}\bar{B_i}) $
    \\
    $B_i$ are independent between each other.
    \\
    $\mathbb{P}(A) = \mathbb{P}(B_1B_2B_3) + \mathbb{P}(\bar{B}_1\bar{B}_2\bar{B}_3) = \frac{1}{4}$
    \\
    $\mathbb{P}(B) = \binom{1}{3}(\frac{1}{2})^3+(\frac{1}{2})^3 = \frac{1}{2}$ 
    \\
    $\mathbb{P}(C) = 1 - \mathbb{P}(A) = \frac{3}{4}$
    \\
    As $A\cap B = $ ``all the child are girls'' : $\mathbb{P}(A\cap B) = \frac{1}{8}$
    and $\mathbb{P}(A)\mathbb{P}(B) = \frac{1}{4}\frac{1}{2} = \frac{1}{8}$ so A and B are independent.
    \\
    So A and B are independent.
    \\
    As $B\cap C = $``there is exactly a boy and 2 girls'' : $\mathbb{P}(B\cap C) = \frac{3}{8}$ and $\mathbb{P}(B)\mathbb{P}(C)= \frac{3}{4}\frac{1}{2}=\frac{3}{8}$
    \subsection*{4.2}
    $C =\bar{A}$ so they can't be independent.

    \subsection*{4.3}
    I will denote the probability of having a boy as $p$ such that : $\forall i\in \llbracket 1;3\rrbracket, \mathbb{P}(B_i)=p$ with $p\in [0;1]$
    \\
    $\mathbb{P}(A) = \mathbb{P}(B_1B_2B_3) + \mathbb{P}(\bar{B}_1\bar{B}_2\bar{B}_3) = p^3 + (1-p)^3$
    \\
    $\mathbb{P}(B) = \binom{1}{3}p(1-p)^2+(1-p)^3 $
    \\
    $\mathbb{P}(A\cap B) = (1-p)^3$ so $\mathbb{P}(A)\mathbb{P}(B)$ 
    is not the same thing for all p after the unicity of the polynomial decomposition.
    The degree for the polynom of $A\cap B$ is $3$ whereas it is 5 for the other so they can't be equal for all p.
    \\
    $\mathbb{P}(C) = 1 - \mathbb{P}(A) = 1 - p^3 -(1-p)^3$
    \\
    $\mathbb{P}(B\cap C) = \binom{3}{1}p(1-p)^2$ again : for a question of unicity of the polynomial decomposition B and C are not independent.
    
    \subsection*{4.4}
    If the family is composed of 4 children : 
    $\mathbb{P}(A) = \frac{1}{8}$, $\mathbb{P}(B) = \frac{5}{16}$, $\mathbb{P}(C) = \frac{7}{8}$, 
    \\
    $\mathbb{P}(A\cap B) = \frac{1}{16}$, $\mathbb{P}(A)\mathbb{P}(B) = \frac{1}{8}\frac{5}{16} \neq \frac{1}{16}$ because $\frac{5}{8} \neq 1$.
    \\
    $\mathbb{P}(B\cap C) = \frac{1}{4}$, $\mathbb{P}(B)\mathbb{P}(C) = \frac{5}{16}\frac{7}{8} \neq \frac{1}{4}$ because $\frac{35}{128}$ is irreducible.
    \\
    The result does not hold  if the family has 4 children.
    
    
    \section*{Exercise 5}
    Suppose $\mathbb{P}(T|G) = \mathbb{P}(G|T)$ : 
    $\mathbb{P}(T|G) = \frac{\mathbb{P}(T\cap G)}{\mathbb{P}(G)}=\frac{\mathbb{P}(G|T)\mathbb{P}(T)}{\mathbb{P}(G)} = \frac{\mathbb{P}(T|G)\mathbb{P}(T)}{\mathbb{P}(G)}$
    \\
    Thus : $1 = \frac{\mathbb{P}(T)}{\mathbb{P}(G)}$ so $\mathbb{P}(T) = \mathbb{P}(G)$
    \\
    Reciprocally suppose $\mathbb{P}(T) = \mathbb{P}(G)$ then $\mathbb{P}(G|T) = \frac{\mathbb{P}(G \cap T)}{\mathbb{P}(T)} =\frac{\mathbb{P}(G \cap T)}{\mathbb{P}(G)} = \mathbb{P}(T|G)$
    \section*{Exercise 6}
    \subsection*{6.1}
    The probability of $As$ is : $\mathbb{P}(As) = \frac{\binom{1}{1}\binom{51}{1}}{\binom{52}{2}}$
    \\
    The probability of $B$ is  : $\mathbb{P}(B) = \frac{\binom{4}{2}}{\binom{52}{2}}$
    \\
    Moreover : The probability of $As$ knowing $B$ is : $\frac{\binom{3}{1}}{\binom{4}{2}}=\frac{1}{2}$ as there is $\binom{4}{2}$ possibilities of having two aces and 3 possibilities of having two aces and the ace of spade
    \\
    $\mathbb{P}(B|As) = \frac{\mathbb{P}(As|B)\mathbb{P}(B)}{\mathbb{P}(As)} = \frac{\frac{1}{2}\frac{\binom{4}{2}}{\binom{52}{2}}}{\frac{\binom{51}{1}}{\binom{52}{2}}} = \frac{3}{51}$.
    \\
    \subsection*{6.2}
    The number of possibilities in A are the total number of possibility minus the number of possibilities where you don't pick any ace.
    $\mathbb{P}(A) = \frac{\binom{52}{2}-\binom{48}{2}}{\binom{52}{2}}$
    \\
    As B is included in A : the probability of $A$ knowing $B$ is 1.
    \\

    $\mathbb{P}(B|A) = \frac{\mathbb{P}(A|B)\mathbb{P}(B)}{\mathbb{P}(A)} = \frac{1*\frac{\binom{4}{2}}{\binom{52}{2}}}{\frac{\binom{52}{2}-\binom{48}{2}}{\binom{52}{2}}} = \frac{1}{33}$

\end{document}