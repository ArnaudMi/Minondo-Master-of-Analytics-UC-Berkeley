\documentclass{article}


\usepackage{array}
\usepackage{amsfonts}
\usepackage{amsmath}
\usepackage{geometry}
\usepackage{stmaryrd}
\usepackage{xcolor}

\definecolor{Zgris}{rgb}{0.95,0.95,0.95}

\geometry{hmargin=2.5cm,vmargin=1.5cm}


\newsavebox{\BBbox}
\newenvironment{DDbox}[1]{
\begin{lrbox}{\BBbox}
    \begin{minipage}{\linewidth}}
{\end{minipage}
\end{lrbox}\noindent\colorbox{Zgris}{\usebox{\BBbox}} \\
[.5cm]}

\title{IEOR240 : Homework 2}
\author{Arnaud Minondo}

\begin{document}
\maketitle
\section*{Problem 1 : True or False}
For the following problem I considered the following linear  optimization problem : 
\begin{equation}
    \min(c^Tx),  \text{ s.t. } x\in\mathbb{R}^n, A\in \mathcal{M}_{m,n}(\mathbb{R}), b\in\mathbb{R}^m, Ax \ge b, x\ge 0_{\mathbb{R}^n}
\end{equation}
\subsection*{1.a}
As the problem in symmetrical is equivalent to the original optimization problem then if there exist an unboundedness certificate on the symmetrical form then the problem is unbounded.
\subsection*{1.b}
If a problem has an optimal solution then its dual also has and let $y$ be the solution of the dual then $y$ is a certificate of boundedness for the primal problem.
\subsection*{1.c}
To know whether a problem is feasable or not you
 need to find an $x$ verifying all the conditions, 
 ie. $x\ge 0$, $Ax \ge b $
 \\
 For example :
 $\max (x_1+x_2), \text{ s.t. } x_1,x_2\ge 0, x_1+x_2<=4$ is feasable because 
 $x=(0,0)$ is a certificate of feasability for this problem.
\subsection*{1.d}
This is true after the strong duality theorem, as $\mathcal{P}$ is feasable and bounded then $\mathcal{D}$ is feasable and bounded and both have optimal solutions.

\subsection*{1.e}
This is not true, if the maximizing/minimizing direction goes in the direction of a border then the problem can be bounded.
\\
For example : $\min x_1+x_2, x_1\ge 0, x_2\ge 0$, the feasable region : $S = (\mathbb{R}^+)^2$ is unbounded but there is an optimal solution which is $x_1=0,x_2=0$.
\subsection*{1.f}
This is true, as the set containing all feasable solution : $S = \{x\in\mathbb{R}^n, Ax\ge b\}$ is convex then if there are two optimal solution : $x_0,x_1$ then $\forall \lambda\in[0,1], \lambda x_0+(1-\lambda)x_1 = \bar{x}\in S $ so is a feasable solution.
\\
Moreover : $c^Tx_1 = c^Tx_0$ so $c^T(x_1-x_0) = 0$ and $c^T\bar{x} = \lambda c^Tx_0+(1-\lambda)c^Tx_1 = c^Tx_1$ so $\bar{x}$ is also optimal.
\subsection*{1.g}
This is false for example : $\max (x_1+x_2), \text{ s.t. } x_1,x_2\ge 0, x_1+x_2<=4$
 is bounded because $y = (1)$  is a certificate of boundedness for this problem.
\subsection*{1.h}
This is false because to be unbounded means that the objective function can be as big/small in case of max/min.
But there are no any x that can be plugged into the objective function so the function can't be as big/small as you wish.
Althought there might exist a certificate for both of boundedness and feasability.
\subsection*{1.i}
It is false because of the Theorem ``either bounded or unbounded''. 
\subsection*{1.j}
This is true because : a max can be changed into a min,
a $\leq$ can be changed by multiplying by $-1$,
an $=$ can be changed into two inequalities,
if a variable $x_i$ is free you can say $x_i=x_i^+-x_i^-$ with $x_i^+,x_i^-\ge0$.


\section*{Problem 2 : Linear Program}
\subsection*{2.a}
The problem in symmetric form is :
\begin{equation}
    \begin{array}{c}
        \min(c^Tx)\\
        \text{s.t. } Ax\ge b, x \ge 0\\
    \end{array}
\end{equation}
Where $A = \left(\begin{array}{ccccc}
    -1&1&-1&0&0\\
    0&-2&2&3&0\\
    1&0&0&0&1\\
\end{array}\right)$, $b = \left(\begin{array}{c}
    -4\\
    2\\
    5\\
\end{array}\right)$ and $c =\left(\begin{array}{c}
    2\\
    -3\\
    3\\
    5\\
    6\\
\end{array}\right)$ 
\\
\subsection*{2.b}
Its dual is : \begin{equation}
    \begin{array}{c}
        \max(b^Ty)\\
        \text{s.t. }y^TA\leq c^T\\
    \end{array}
\end{equation}
\subsection*{2.c}
My model file : pb2.mod
\\
\begin{DDbox}{\linewidth}
    \begin{verbatim}
        param n;
        param p;
        
        param c{i in 1..n};
        param b{i in 1..p};
        
        param A{j in 1..p, i in 1..n};
        
        var u{i in 1..n};
        
        minimize object: sum{i in 1..n}c[i]*u[i];
        
        s.t. constraints1{j in 1..p}: sum{i in 1..n}A[j,i]*u[i]>=b[j];
        s.t. constraints2{i in 1..n}: u[i]>=0;
    \end{verbatim}
\end{DDbox}
\\
\newpage
My data file : pb2.dat
\\
\begin{DDbox}{\linewidth}
    \begin{verbatim}
        param n := 5;
        param p := 3;
        
        param b := 
        1 -4
        2 2
        3 5;
        
        param A :
           1  2  3  4  5=
        1 -1  1 -1  0  0
        2  0 -2  2  3  0
        3  1  0  0  0  1;
        
        param c:=
        1 2
        2 -3
        3 3
        4 5
        5 6; 
    \end{verbatim}
\end{DDbox}
\\
My run file : pb2.run
\\
\begin{DDbox}{\linewidth}
    \begin{verbatim}
        reset;

        model pb2.mod;
        
        data pb2.dat;
        
        option solver cplex;
        solve;
        
        display object, u;
    \end{verbatim}
\end{DDbox}
\\
The solution for this problem is : $x_1 = 5, x_2 = 1, x_3 = 1.333, x_4 = 0$ with the $\min = 13,6667$
\\
The dual problem solution is : $y = (\frac{1}{3},\frac{5}{3}, \frac{7}{3})$ 
\\
The data file is the same.
The model file : pb2\_dual.mod 
\\
\begin{DDbox}{\linewidth}
    \begin{verbatim}
        param p;
        param n;
        
        param A{i in 1..p,j in 1..n};
        
        param b{i in 1..p};
        
        param c{j in 1..n};
        
        var y{i in 1..p};
        
        
        maximize object: sum{i in 1..p}y[i]*b[i];
        
        s.t. constraints{j in 1..n} :sum{i in 1..p}y[i]*A[i,j] <=c[j];
    \end{verbatim}
\end{DDbox}
\\
\newpage
The run file : pb2\_dual.run
\\
\begin{DDbox}{\linewidth}
    \begin{verbatim}
        reset;

        model pb2_dual.mod;
        data pb2_dual.dat;
        
        option solver cplex;
        
        solve;
        
        display object, y;
    \end{verbatim}
\end{DDbox}
\section*{Problem 3 : General Problem}
\subsection*{(a)}
($\mathcal{P}$) is feasable because $x =0_{\mathbb{R}^n}$ is a feasable solution.
 As ($\mathcal{P}$) is feasable its dual can't be unbounded, it is either unfeasable if $(\mathcal{P})$ is unbounded and is feasable in the other case.
\subsection*{(b)}
($\mathcal{D}$) objective function is : $y^T0_{\mathbb{R}^m}$ so it's objective value will be 0.
 Moreover, $x=0_{\mathbb{R}^n}$ verifies $c^Tx=y^T0_{\mathbb{R}^m}$ which means x is optimal.
 The objective value function is then 0.
\section*{Problem 4 :}
The problem is feasable as $x=0_{\mathbb{R}^n}$ is a feasable solution. Moreover it is bounded as $c \ge 0$ and $x \ge 0$ so $c^Tx \ge 0$.\\
 Thus $ \min(c^Tx)\ge 0$ and the problem is bounded. As it is feasable and bounded it must have an optimal solution.
\end{document}